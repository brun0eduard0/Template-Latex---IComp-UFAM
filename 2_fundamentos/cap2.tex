% ---
% Capítulo 2
% ---
\chapter{Fundamentos}

\lettrine[lines=3]{E}{} ste é o primeiro capítulo da parte central do trabalho, isto é, o desenvolvimento, a parte mais extensa de todo o trabalho. Geralmente o desenvolvimento é dividido em capítulos, cada um com seções e subseções, cujo tamanho e número de divisões variam em função da natureza do conteúdo do trabalho.

	Em geral, a parte de desenvolvimento é subdividida em três capítulos:

	\begin{itemize}
		\item \textit{referencial ou embasamento teórico} – texto no qual se deve apresentar os aspectos teóricos, isto é, os conceitos utilizados e a definição dos mesmos; nesta parte faz-se a revisão de literatura sobre o assunto, resumindo-se os resultados de estudos feitos por outros autores, cujas obras citadas e consultadas devem constar nas referências;
	
		\item \textit{metodologia do trabalho ou procedimentos metodológicos} – deve constar o instrumental, os métodos e as técnicas aplicados para a elaboração do trabalho;
	
		\item \textit{resultados} – devem ser apresentados, de forma objetiva, precisa e clara, tanto os resultados positivos quanto os negativos que foram obtidos com o desenvolvimento do trabalho, sendo feita uma discussão que consiste na avaliação circunstanciada, na qual se estabelecem relações, deduções e generalizações.
	\end{itemize}

	É recomendável que o número total de páginas referente à parte de desenvolvimento não ultrapasse 60 (sessenta) páginas.

	\section{Seção 1}

		Teste de figura: são 2 métodos para fazê-lo

		\begin{figure}[!ht]
			\begin{center}
			    \includegraphics[scale=0.5]{abntex2/ufam-logo}
			\end{center}
			\caption{\label{fig_grafico}Logo da UFAM Retirado da Internet}
		\end{figure}
		
		\begin{figure}[h]
	        \begin{center}
		        \includegraphics[width=0.2\textwidth]{abntex2/ufam-logo}
	        \end{center}
	        \caption{\label{fig:logoUfam}Logo da UFAM Retirado da Internet}
        \end{figure}
		
		
		Continuação do texto.
		
	\section{Seção 2}
	
		Referenciamento da figura inserida na seção anterior: 2.1
		
	\section{Seção 3}
	
		Seção 3
		
	\section{Abreviaturas e Simbolos}
	
	 Para utilizar abreviaturas e simbolos, basta utilizar o comando barra \ combinado com ac ou acl.
	 
	 Abreviaturas e a versão por extenso.
	 
	 AC \ac{abntex}.
	 
	 ACL \acl{abntex}.
	 
	 Simbolos e a versão por extenso.
	 
	 AC  \ac{lambda}.
	 
	 ACL \acl{lambda}.

