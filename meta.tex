
% ---
% Pacotes básicos
% ---
\usepackage{bookmark}				% Usa a fonte Bookman Old Style
\usepackage[T1]{fontenc}			% Selecao de codigos de fonte.
\usepackage[utf8]{inputenc}		% Codificacao do documento (conversão automática dos acentos)
\usepackage{color}				% Controle das cores
\usepackage{graphicx}			% Inclusão de gráficos
\usepackage{microtype} 			% para melhorias de justificação

\usepackage[brazilian,hyperpageref]{backref}	 % Paginas com as citações na bibl
\usepackage[alf]{abntex2cite}	% Citações padrão ABNT
\usepackage{float}
\usepackage{url}
\usepackage{enumerate}
\usepackage{siunitx}
\usepackage{setspace}
\usepackage[euler]{textgreek}

%%%%%%%%% NOVO
\usepackage{hyperref}
\hypersetup{
    colorlinks=true,
    linkcolor=blue,
    citecolor=blue,
    filecolor=magenta,      
    urlcolor=blue,
    bookmarksopen=true,
}
\urlstyle{same}

%%%   letra capitular
\usepackage{lettrine}
\usepackage{palatino}

%%%    tabelas - configuração
\usepackage{booktabs}
\usepackage{siunitx}

\usepackage{lscape}


%%% lista de abreviações e siglas automática 
\usepackage{acro}

%%%  LISTA DE ABREVIAÇÕES
% probably a good idea for the nomenclature entries:

\acsetup{first-style=short}
%Padrão para a lista de abreviações
% class `abbrev': abbreviations:

%NOTE QUE SOMENTE OS QUE SÃO CHAMADOS NO DECORRER DO TEXTO IRÃO APARECER NA LISTA DE ABREVIATURAS E SIMBOLOS
%PADRÃO PARA ABREVIATURA
%\DeclareAcronym{}{
%	short = ,
%	long  = \textit{},
%	class = abbrev
%}

%PADRÃO PARA SIMBOLO
%\DeclareAcronym{}{
%	short = ,
%	long  = \textit{},
%   sort = ,
%	class = nomencl
%}
% use a página a seguir para coletar simbolos para pôr no latex
% https://texblog.org/2012/03/15/greek-letters-in-text-without-changing-to-math-mode/

\DeclareAcronym{sbc}{
	short = SBC,
	long  = Sociedade Brasileira de Computação,
	class = abbrev
}

\DeclareAcronym{abnt}{
	short = \textit{ABNT},
	short-plural = \textit{s},
	long  = \textit{Associação Brasileira de Normas Técnicas},
	class = abbrev
}

\DeclareAcronym{abntex}{
	short = \textit{abnTeX},
	short-plural = \textit{s},
	long  = \textit{ABsurdas Normas para TeX},
	class = abbrev
}

%\DeclareAcronym{}{
%	short = ,
%	long  = \textit{},
%	class = abbrev
%}

%%%  LISTA DE SIMBOLOS
%Padrão para lista de símbolos
% class `nomencl': simbolos
\DeclareAcronym{megaByte}{
	short = MB ,
	long  = é uma unidade de medida de informação que equivale a 1000000 bytes,
	sort  = MB,
	class = nomencl
}

% para simbolos matemáticos, utilize o padrão a seguir, pois não necessita pôr o texto em modo matemático
% https://texblog.org/2012/03/15/greek-letters-in-text-without-changing-to-math-mode/

\DeclareAcronym{lambda}{
	short =  \textlambda ,
	long  = comprimento de onda,
	sort  = \textlambda,
	class = nomencl
}


%%%%%% estilo do capítulo
%%%%%% escolha 1 e tire o comentário
%% acesse a página e escolha
%%http://ctan.math.washington.edu/tex-archive/info/latex-samples/MemoirChapStyles/MemoirChapStyles.pdf

%% primeiro
%\chapterstyle{madsen}

%% segundo
%\chapterstyle{southall}

%% terceiro
%\chapterstyle{Ger}

%% quarto
%\chapterstyle{verville}

%% quinto
\setlength\midchapskip{10pt}
\makechapterstyle{VZ23}{
    \renewcommand\chapternamenum{}
    \renewcommand\printchaptername{}
    \renewcommand\chapnumfont{\Huge\bfseries\centering}
    \renewcommand\chaptitlefont{\Huge\scshape\centering}
    \renewcommand\afterchapternum{%
        \par\nobreak\vskip\midchapskip\hrule\vskip\midchapskip}
    \renewcommand\printchapternonum{%
        \vphantom{\chapnumfont \thechapter}
        \par\nobreak\vskip\midchapskip\hrule\vskip\midchapskip}
}
\chapterstyle{VZ23}

%%%%%%%%%%%%%

\titulo{Nome da tese, dissertação ou TCC}
\autor{Nome do aluno}
\local{Manaus - AM}
\data{mês e ano}
\orientador[Orientador(a)]{Fulano de Tal, Dr.}%Nome e titulação do(a) professor(a) orientador(a)}
\instituicao{%
  Universidade Federal do Amazonas - UFAM
  \par
  Instituto de Computação- IComp}
\curso{%
  Programa Pós-Graduação em Informática - PPGI}
\tipotrabalho{Monografia}
% O preambulo deve conter o tipo do trabalho, o objetivo, 
% o nome da instituição e a área de concentração 

\preambulo{Exame de Qualificação submetido à avaliação, como requisito parcial, para a obtenção do título de Mestre em Informática no Programa de Pós-Graduação em Informática, Instituto de Computação.}


% Informações do PDF
\makeatletter
\hypersetup{
    	%pagebackref=true,
	pdftitle={\@title}, 
	pdfauthor={\@author},
    	pdfsubject={\imprimirpreambulo},
    pdfcreator={LaTeX with abnTeX2},
	pdfkeywords={abnt}{latex}{abntex}{abntex2}{trabalho acadêmico},
	bookmarksdepth=4
}
\makeatother

% --- 
% Espaçamentos entre linhas e parágrafos 
% --- 

% O tamanho do parágrafo é dado por:
\setlength{\parindent}{1.3cm}

% Controle do espaçamento entre um parágrafo e outro:
%\setlength{\parskip}{0.2cm}  % tente também \onelineskip

%\setbeforesecskip{3em}
%\setbeforesubsecskip{3em}

% ---
% compila o indice
% ---
\makeindex