% ---
% Capítulo 3
% ---
\chapter{Capítulo 3}

Algumas regras devem ser observadas na redação da monografia:.


	\begin{itemize}
		\item ser claro, preciso, direto, objetivo e conciso, utilizando frases curtas e evitando ordens inversas desnecessárias;
		
		\item construir períodos com no máximo duas ou três linhas, bem como parágrafos com cinco linhas cheias, em média, e no máximo oito (ou seja, não construir parágrafos e períodos muito longos, pois isso cansa o(s) leitor(es) e pode fazer com que ele(s) percam a linha de raciocínio desenvolvida);
		
		\item a simplicidade deve ser condição essencial do texto; a simplicidade do texto não implica necessariamente repetição de formas e frases desgastadas, uso exagerado de voz passiva (como será iniciado, será realizado), pobreza vocabular etc. Com palavras conhecidas de todos, é possível escrever de maneira original e criativa e produzir frases elegantes, variadas, fluentes e bem alinhavadas;
		
		\item adotar como norma a ordem direta, por ser aquela que conduz mais facilmente o leitor à essência do texto, dispensando detalhes irrelevantes e indo diretamente ao que interessa, sem rodeios (verborragias);

		\item não começar períodos ou parágrafos seguidos com a mesma palavra, nem usar repetidamente a mesma estrutura de frase;

		\item desprezar as longas descrições e relatar o fato no menor número possível de palavras;
		
		\item recorrer aos termos técnicos somente quando absolutamente indispensáveis e nesse caso colocar o seu significado entre parênteses (ou seja, não se deve admitir que todos os que lerão o trabalho já dispõem de algum conhecimento desenvolvido no mesmo);
		
		\item dispensar palavras e formas empoladas ou rebuscadas, que tentem transmitir ao leitor mera ideia de erudição;

		\item não perder de vista o universo vocabular do leitor, adotando a seguinte regra prática: nunca escrever o que não se diria;

		\item usar termos coloquiais ou de gíria com extrema parcimônia (ou mesmo nem serem utilizados) e apenas em casos muito especiais, para não darem ao leitor a ideia de vulgaridade e descaracterizar o trabalho;

		\item ser rigoroso na escolha das palavras do texto, desconfiando dos sinônimos perfeitos ou de termos que sirvam para todas as ocasiões;

		\item em geral, há uma palavra para definir uma situação;

		\item encadear o assunto de maneira suave e harmoniosa, evitando a criação de um texto onde os parágrafos se sucedem uns aos outros como compartimentos estanques, sem nenhuma fluência entre si;

		\item ter um extremo cuidado durante a redação do texto, principalmente com relação às regras gramaticais e ortográficas da língua;

		\item geralmente todo o texto é escrito na forma impessoal do verbo, não se utilizando, portanto, de termos em primeira pessoa, seja do plural ou do singular.

	\end{itemize}

	Continuação.
	
	\section{Tabelas}
	
		Teste de uma tabela:

		\begin{table}[htbp]
			\caption{Tabela sem sentido.}
			\label{tabela-ssentido}
			\begin{center}
			\begin{tabular}{|c|c|}
				\hline
				Título Coluna & Título Coluna \\
				1 & 2 \\
				\hline
				X & Y \\
				\hline
				X & W \\
				\hline
			\end{tabular}
			\end{center}
		\end{table}
		
		\begin{table}[H]
        	\centering
        	\begin{tabular}{|c|c|c|}
        		\hline
        		\textbf{LAN sem fio} & \textbf{Frequência} & \textbf{Taxa de Transmissão}  \\ \hline
        		\textbf{802.11b} & 2,4 a 2,485 GHz& 11 Mbps\\ \hline
        		\textbf{802.11g} & 2,4 GHz e 5 GHz& 54 Mbps \\ \hline
        		\textbf{802.11n} & 2,4 GHz e 5 GHz & 200 Mbps a 600 Mbps \\ \hline
        		\textbf{LoRa} & 868, 433, 915, 902 e 928 MHz & 50 Kbps \\ \hline
        	\end{tabular}
        	\caption{\label{table:transmissaoSemFio}Comparação entre os padrões de comunicação sem fio}
        \end{table}

        \begin{table}[h]
        	\centering
        	\begin{tabular}{p{7cm}S[table-format=0.2]S[table-format=0.2]S[table-format=0.2]S[table-format=0.2]S[table-format=0.2]S[table-format=0.2]S[table-format=0.2]}
        		\toprule	
        		2019 & {Ago} & {Set} & {Out} & {Nov} & {Dez} & {Jan} & {Fev}
        		\\
        		\midrule
        		Revisão bibliográfica  & X &  &  &  &  &  &  \\ 
        		Escrever e revisar dissertação  & X & X & X & X & X &  &  \\
        		Preparação dos experimentos & X & X & X & X &  &  &  \\ 
        		Experimentos e análise dos resultados  & X & X & X & X & X &  &  \\ 
        		Publicação de Resultados &  &  &  &  & X & X & X \\
        		\bottomrule
        	\end{tabular}
        	\caption{\label{table:cronogramaT}Cronograma.}
        \end{table}		
		
        \begin{table}[H]
        	\centering
        	\begin{tabular}{p{0.8cm}S[table-format=0.1]S[table-format=0.1]S[table-format=0.1]S[table-format=0.1]S[table-format=0.1]}
        		\toprule
        		{\textbf{Nº}} & {\textbf{Ano}} & {\textbf{Mula}} & {\textbf{Network Coding}} & {\textbf{Routing}} & {\textbf{Simulador}}
        		\\
        		\midrule
        		{[1]} & {2012} & {Proteus} & {XOR} & \textit{store-and-forward} & {aplicação real}   \\
        		{[2]} & {2016} & {Ônibus} & {XOR} & \textit{store-and-forward} & \textit{The ONE} \\ 
        		{[3]} & {2016} & {Barco} & {XOR} & {ER, EOR, FCR, SWR} & \textit{The ONE}   \\
        		\bottomrule
        	\end{tabular}
        	\caption{\label{table:trabalhoMulaNC}Relação dos trabalhos sobre Mula de Dados com NC}	
        \end{table}
	
	\section{Seção 2}
	
		Seção 2.

	\section{Seção 3}
	
		Seção 3.
		