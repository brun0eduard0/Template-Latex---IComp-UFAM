\noindent

% Seleciona o idioma do documento (conforme pacotes do babel)
%\selectlanguage{english}
\selectlanguage{brazil}

% Retira espaço extra obsoleto entre as frases.
\frenchspacing

% ----------------------------------------------------------
% ELEMENTOS PRÉ-TEXTUAIS
% ----------------------------------------------------------
% \pretextual

% ---
% Capa
% ---
\imprimircapa
% ---

% ---
% Folha de rosto
% (o * indica que haverá a ficha bibliográfica)
% ---
\imprimirfolhaderosto*
% ---

% ---
% Inserir folha de aprovação
% ---

% Isto é um exemplo de Folha de aprovação, elemento obrigatório da NBR
% 14724/2011 (seção 4.2.1.3). Você pode utilizar este modelo até a aprovação
% do trabalho. Após isso, substitua todo o conteúdo deste arquivo por uma
% imagem da página assinada pela banca com o comando abaixo:
%
% \includepdf{folhadeaprovacao_final.pdf}
%
\begin{folhadeaprovacao}
	\parindent=0pt
	\setlength{\ABNTEXsignskip}{1.5cm}

	Monografia de Graduação sob o título <\textit{Título da monografia}> apresentada por <Nome do aluno> e aceita pelo Instituto de Computação da Universidade Federal do Amazonas, sendo aprovada por todos os membros da banca examinadora abaixo especificada:

	\assinatura{\fontsize{12}{15}\selectfont Titulação e nome do(a) orientador(a) \\ \fontsize{11}{15}\selectfont \imprimirorientadorRotulo~ \\ {\fontsize{10}{12}\selectfont Departamento \par Universidade}}
	\vspace{1cm}
	\assinatura{\fontsize{12}{15}\selectfont Titulação e nome do(a) membro da banca examinadora \\ \fontsize{11}{15}\selectfont Co-orientador(a), se houver \\ {\fontsize{10}{12}\selectfont Departamento \par Universidade}}
	\vspace{1cm}
	\assinatura{Titulação e nome do membro da banca examinadora \\ {\fontsize{10}{12}\selectfont Departamento \par Universidade}}
	\vspace{1cm}
	\assinatura{Titulação e nome do membro da banca examinadora \\ {\fontsize{10}{12}\selectfont Departamento \par Universidade}}
	\vfill
      
	\begin{center}
		\fontsize{12}{15}\selectfont
		\vspace*{0.5cm}
		\imprimirlocal, data de aprovação (por extenso).
		\vspace*{1cm}
	\end{center}
  
\end{folhadeaprovacao}

% ---
% Dedicatória
% ---
\begin{dedicatoria}
   \vspace*{\fill}
   \noindent
   \leftskip=5cm

   Homenagem que o autor presta a uma ou mais pessoas.

   \vspace{5cm}
\end{dedicatoria}
% ---

% ---
% Agradecimentos
% ---
\begin{agradecimentos}

Agradecimentos dirigidos àqueles que contribuíram de maneira relevante à elaboração do trabalho, sejam eles pessoas ou mesmo organizações.

\end{agradecimentos}

% ---
% Epígrafe
% ---
\begin{epigrafe}
    \vspace*{\fill}
	\begin{flushright}
		\textit{Citação}

		Autor
	\end{flushright}\vspace{4cm}
\end{epigrafe}
% ---

% ---
% RESUMOS
% ---

% resumo em português
\setlength{\absparsep}{18pt} % ajusta o espaçamento dos parágrafos do resumo
\begin{resumo}

 	O resumo deve apresentar de forma concisa os pontos relevantes de um texto, fornecendo uma visão rápida e clara do conteúdo e das conclusões do trabalho. O texto, redigido na forma impessoal do verbo, é constituído de uma sequência de frases concisas e objetivas e não de uma simples enumeração de tópicos, não ultrapassando 500 palavras, seguido, logo abaixo, das palavras representativas do conteúdo do trabalho, isto é, palavras-chave e/ou descritores. Por exemplo, deve-se evitar, na redação do resumo, o uso de fórmulas, equações, diagramas e símbolos, optando-se, quando necessário, pela transcrição na forma extensa, além de não incluir citações bibliográficas.

 \textit{Palavras-chave}: Palavra-chave 1, Palavra-chave 2, Palavra-chave 3.

\end{resumo}

% resumo em inglês
\begin{resumo}[Abstract]
 \begin{otherlanguage*}{english}
   This is the english abstract.

   \vspace{\onelineskip}
 
   \noindent 
   \textit{Keywords}: Keyword 1, Keyword 2, Keyword 3.
 \end{otherlanguage*}
\end{resumo}

\frontmatter

% ---
% inserir lista de figuras
% ---
\pdfbookmark[0]{\listfigurename}{lof}
\listoffigures*
\cleardoublepage
% ---

% ---
% inserir lista de tabelas
% ---
\pdfbookmark[0]{\listtablename}{lot}
\listoftables*
\cleardoublepage
% ---

% ---
% inserir lista de abreviaturas e siglas
% ---
\begin{siglas}
\end{siglas}
% ---

% ---
% inserir lista de símbolos
% ---
\begin{simbolos}
\end{simbolos}
% ---

% ---
% inserir o sumario
% ---


\pdfbookmark[0]{\contentsname}{toc}
\tableofcontents*
\cleardoublepage
% ---

\textual