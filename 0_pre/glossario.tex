%%%  LISTA DE ABREVIAÇÕES
% probably a good idea for the nomenclature entries:

\acsetup{first-style=short}
%Padrão para a lista de abreviações
% class `abbrev': abbreviations:

%NOTE QUE SOMENTE OS QUE SÃO CHAMADOS NO DECORRER DO TEXTO IRÃO APARECER NA LISTA DE ABREVIATURAS E SIMBOLOS
%PADRÃO PARA ABREVIATURA
%\DeclareAcronym{}{
%	short = ,
%	long  = \textit{},
%	class = abbrev
%}

%PADRÃO PARA SIMBOLO
%\DeclareAcronym{}{
%	short = ,
%	long  = \textit{},
%   sort = ,
%	class = nomencl
%}
% use a página a seguir para coletar simbolos para pôr no latex
% https://texblog.org/2012/03/15/greek-letters-in-text-without-changing-to-math-mode/

\DeclareAcronym{sbc}{
	short = SBC,
	long  = Sociedade Brasileira de Computação,
	class = abbrev
}

\DeclareAcronym{abnt}{
	short = \textit{ABNT},
	short-plural = \textit{s},
	long  = \textit{Associação Brasileira de Normas Técnicas},
	class = abbrev
}

\DeclareAcronym{abntex}{
	short = \textit{abnTeX},
	short-plural = \textit{s},
	long  = \textit{ABsurdas Normas para TeX},
	class = abbrev
}

%\DeclareAcronym{}{
%	short = ,
%	long  = \textit{},
%	class = abbrev
%}

%%%  LISTA DE SIMBOLOS
%Padrão para lista de símbolos
% class `nomencl': simbolos
\DeclareAcronym{megaByte}{
	short = MB ,
	long  = é uma unidade de medida de informação que equivale a 1000000 bytes,
	sort  = MB,
	class = nomencl
}

% para simbolos matemáticos, utilize o padrão a seguir, pois não necessita pôr o texto em modo matemático
% https://texblog.org/2012/03/15/greek-letters-in-text-without-changing-to-math-mode/

\DeclareAcronym{lambda}{
	short =  \textlambda ,
	long  = comprimento de onda,
	sort  = \textlambda,
	class = nomencl
}